\documentclass[11pt, oneside]{article}   	% use "amsart" instead of "article" for AMSLaTeX format
\usepackage{geometry}                		% See geometry.pdf to learn the layout options. There are lots.
\geometry{letterpaper}                   		% ... or a4paper or a5paper or ... 
%\geometry{landscape}                		% Activate for rotated page geometry
%\usepackage[parfill]{parskip}    		% Activate to begin paragraphs with an empty line rather than an indent
\usepackage{graphicx}				% Use pdf, png, jpg, or eps§ with pdflatex; use eps in DVI mode
								% TeX will automatically convert eps --> pdf in pdflatex		
\usepackage{amssymb}
\usepackage{amsmath}

%SetFonts

%SetFonts


\title{Brief Article}
\author{The Author}
%\date{}							% Activate to display a given date or no date

\begin{document}
\maketitle
%\section{}
%\subsection{}
\abstract{
The purpose of this analysis:
Given $N$ pointings determine the placement of fibers that optimizes the determination of the asymptotic velocity
of the rotation curve of a spiral galaxy.
The procedure: Determine the combination of positions that minimize the uncertainty of the asymptotic velocity
as estimated from Fisher Matrix calculations.
}

\section{Data}
The data are composed of $N$ measurements.  Each have a
\begin{itemize}
\item $\hat{\theta}$ - the angular coordinate where we think the fiber was positioned.
\item $\hat{v}$ - velocity measurement.
\end{itemize}
There are a total of $2N$ measurements.
Measurements are notated by the hat.

The velocity measurement uncertainties are given by $\sigma_v$.  
This uncertainty is taken to be the proportional to the inverse of the signal-to-noise of the flux measurement of the feature, that is
\begin{align*}
\sigma_v & \propto \frac{\sqrt{S + B}}{S},
\end{align*}
where $S$ and $B$ are the source and background fluxes respectively.
For $S$ we use the light of the spiral arm, which according to Wikipedia is well described by an exponential profile,
\begin{equation}
S \propto \exp{\left(-b \frac{|\theta|}{R_e}\right)},
\end{equation}
where $b \approx 5/3$ and $R_e$ is the half-light radius.

When the signal to noise gets poor enough a velocity measurement cannot even be obtained, meaning that there 
is a maximum angular separation from the core $\theta^{\text{max}}$  with a velocity  measurement
that has the largest possible uncertainty $ \sigma_v^\text{max}$.  Anticipating a Fisher matrix analysis, which does not accommodate the
discrete dropout of data, we take
\begin{equation}
\sigma_v = 
	\begin{cases}
	 \sigma_v^\text{max} \exp{\left(b \frac{|\theta|-\theta^\text{max}}{R_e}\right)} \sqrt{\frac{\exp{\left(-b \frac{|\theta|}{R_e}\right)}+ B'}{\exp{\left(-b \frac{\theta^\text{max}}{R_e}\right)}+B'}} & |{\theta}| < \theta^{\text{max}}\\
	 \sigma_v^\text{max} \exp{\left(c \frac{|\theta|-\theta^\text{max}}{R_e}\right)} &  |{\theta}| \ge \theta^{\text{max}} 
	 \end{cases}.
\end{equation}
In this note, we take $c=5>b$.
There are three parameters in the noise model of a galaxy, the maximum possible uncertainty
$\sigma_v^\text{max}$, the radius where that occurs $\theta^\text{max}/R_e$, and the relative flux
of the source and background $B'$.


\section{Model}
The true positions of the measurements are assumed to be
\begin{equation}
\theta \sim \mathcal{N}(\hat{\theta},\sigma_\theta),
\end{equation}
where $\sigma_\theta$ is the positioning error.

The velocity $v$ along the semi-major axis is modeled as a hyperbolic tangent
\begin{equation}
f=v_\infty \tanh{\left(\frac{\theta - \theta_0}{a}\right)} + v_0,
\end{equation}
where the model parameters are
\begin{itemize}
\item $v_\infty$ - Asymptotic rotation velocity.
\item $\theta$ - Coordinate of the measurement
\item $\theta_0$ - Coordinate of the galaxy centroid. 
\item $a$ - Galaxy length along the semi-major axis.
\item $v_0$ - the nominal redshift of the galaxy
\end{itemize}
There are a total of $4+N$ parameters.

The elements of the Fisher Matrix for the above model is given in \S\ref{sec:fisher}.

\section{Results for DESI}
DESI is interested in measuring Tully-Fisher distances of large spiral galaxies. Each of these galaxies will have at
least 4 measurements, one of which is at the core of the galaxy.  
Of primary interest is the uncertainty in the asymptotic velocity $\sqrt{F^{-1}_{v_\infty v_\infty}}$.
We therefore determine the placement of the three other fibers by minimizing this uncertainty.

Before moving forward, it is worth mentioning that the optimal solution for four free fibers does not have one of them
at the core.  In addition, we are ultimately interested in the peculiar velocity of the galaxy meaning that
the uncertainty in $v_0$ is also relevant.  However, the fiber placements that optimize the galaxy redshift and rotation velocities differ.
As we proceed, we take that a core measurement is immutable and that  the uncertainties in the rotation velocity
dominate the peculiar velocity error budget.

The optimal fiber placement depends the specific details of the galaxy: its signal-to-noise
as described by the maximum radius where a rotation velocity can be measured $\theta^\text{max}$,
 the length-scale of its rotation curve $a$, and its size relative
to the fiber positioning uncertainty $\sigma_\theta$ (all considered in units of half-light radius $R_e$).

The optimal fiber placements as a function of the angular extent where velocities can be measured
$\theta^\text{max}$ for fixed values of $a$, $\sigma_\theta$,
and $B'=0$ (source noise-dominated) are shown in Figure~\ref{fig:theta_max}.  When rotation velocities can be only measured
out to small radii, small $\theta^\text{max}$, one of the fibers is best placed at the that extremum to provide
as much leverage as possible in measuring the asymptote of the rotation curve.  Above a certain value
of  $\theta^\text{max}$ the fiber placements are the same, the optimal solutions
settling on the fixed velocity signal-to-noises at those positions.

\begin{figure}[htbp] %  figure placement: here, top, bottom, or page
   \centering
   \includegraphics[width=4in]{../src/vary_theta_max.pdf} 
   \caption{Optimal placements of three fibers, with one fixed at the core, as a function of $\theta^\text{max}$ for fixed values of $a=4$, $\sigma_\theta=0.01$,
and $B'=0$. One of the four fibers is fixed at $\theta=0$.}
   \label{fig:theta_max}
\end{figure}

The optimal fiber placements as a function of the scale of the rotation curve
$a$ for fixed values of $\theta^\text{max}$, $\sigma_\theta$,
and $B'=0$ are shown in Figure~\ref{fig:a}. 
The value of $\theta^\text{max}$ is large enough to give velocity measurements over most of the galaxy and so represents the asymptotic solution discussed in the previous example.
A broader angular extent is preferred for broader rotation curves and, within the phase space shown, there is no asymptotic fixed-position 
solution.
The value of $a$ will not be known before the measurement is made, either an independent estimate of $a$ or
an effective value for the population could be used.

\begin{figure}[htbp] %  figure placement: here, top, bottom, or page
   \centering
   \includegraphics[width=4in]{../src/vary_a.pdf} 
   \caption{Optimal placements of three fibers, with one fixed at the core, as a function of $a$ for fixed values of $\theta^\text{max}=4$, $\sigma_\theta=0.01$,
and $B'=0$. One of the four fibers is fixed at $\theta=0$.}
   \label{fig:a}
\end{figure}

The optimal fiber placements as a function of the size of the galaxy in terms of the pointing error
$\sigma_\theta$  for fixed values of $\theta^\text{max}$, $a$,
and $B'=0$ are shown in Figure~\ref{fig:sigma_theta}. 
There is a slight dependence on the galaxy size, with smaller galaxies preferring relatively higher radius fiber placement.
 
\begin{figure}[htbp] %  figure placement: here, top, bottom, or page
   \centering
   \includegraphics[width=4in]{../src/vary_sigma_theta.pdf} 
   \caption{Optimal placements of three fibers, with one fixed at the core, as a function of $\sigma_\theta=0.01$ for fixed values of  $\theta^\text{max}=4$,  $a=4$,
and $B'=0$. One of the four fibers is fixed at $\theta=0$.}
   \label{fig:sigma_theta}
\end{figure}

\appendix
\section{Fisher Matrix Elements}
\label{sec:fisher}
The corresponding partial derivatives for the model prediction for measurement $i$ are
\begin{itemize}
\item $\partial f/ \partial v_\infty =  \tanh{\left(\frac{\theta_i - \theta_0}{a}\right)}$
\item $\partial f/ \partial\theta_j = \frac{v_\infty }{a} \text{sech}^2{\left(\frac{\theta_i - \theta_0}{a}\right)} \delta^D_{ij}$ 
\item $\partial f/ \partial\theta_0 = -\frac{v_\infty }{a} \text{sech}^2{\left(\frac{\theta_i - \theta_0}{a}\right)}$ 
\item $\partial f/ \partial a =-\frac{v_\infty }{a^2} (\theta_i - \theta_0)\text{sech}^2{\left(\frac{\theta_i - \theta_0}{a}\right)}$
\item $\partial f/ \partial v_0 =1$
\end{itemize}

The Fisher matrix elements are then
\begin{itemize}
\item $F_{v_\infty v_\infty} = \sum_i \sigma_{v_i}^{-2}  \tanh^2{\left(\frac{\theta_i - \theta_0}{a}\right)} $
\item $F_{v_\infty \theta_i}  = \frac{v_\infty }{a} \sigma_{v_i}^{-2} \tanh{\left(\frac{\theta_i - \theta_0}{a}\right)} \text{sech}^2{\left(\frac{\theta_i - \theta_0}{a}\right)}  $ 
\item $F_{v_\infty \theta_0}  =  -\frac{v_\infty }{a} \sum_i  \sigma_{v_i}^{-2}  \tanh{\left(\frac{\theta_i - \theta_0}{a}\right)} \text{sech}^2{\left(\frac{\theta_i - \theta_0}{a}\right)} $ 
\item $F_{v_\infty a}  =  -\frac{v_\infty }{a^2} \sum_i  (\theta_i - \theta_0) \sigma_{v_i}^{-2}  \tanh{\left(\frac{\theta_i - \theta_0}{a}\right)} \text{sech}^2{\left(\frac{\theta_i - \theta_0}{a}\right)} $ 
\item $F_{v_\infty v_0} = \sum_i \sigma_{v_i}^{-2}  \tanh{\left(\frac{\theta_i - \theta_0}{a}\right)} $
\item $F_{\theta_i \theta_j}  = ( \frac{v_\infty^2 }{a^2}  \sigma_{v_i}^{-2}\text{sech}^4{\left(\frac{\theta_i - \theta_0}{a}\right)} + \sigma_\theta^{-2}) \delta^D_{ij}$ 
\item $F_{\theta_i \theta_0}  =  - \frac{v_\infty^2}{a^2}  \sigma_{v_i}^{-2}\text{sech}^4{\left(\frac{\theta_i - \theta_0}{a}\right)} $ 
\item $F_{\theta_i a}  =  - \frac{v_\infty^2}{a^3}   (\theta_i - \theta_0) \sigma_{v_i}^{-2}\text{sech}^4{\left(\frac{\theta_i - \theta_0}{a}\right)} $ \item $F_{\theta_i v_0}  =   \frac{v_\infty}{a} \sigma_{v_i}^{-2}\text{sech}^2{\left(\frac{\theta_i - \theta_0}{a}\right)} $ 
\item $F_{\theta_0 \theta_0}  =   \frac{v_\infty^2}{a^2} \sum_i \sigma_{v_i}^{-2}\text{sech}^4{\left(\frac{\theta_i - \theta_0}{a}\right)} $ 
\item $F_{\theta_0 a}  =   \frac{v_\infty^2}{a^3} \sum_i  (\theta_i - \theta_0)  \sigma_{v_i}^{-2}\text{sech}^4{\left(\frac{\theta_i - \theta_0}{a}\right)} $
\item $F_{\theta_0 v_0}  =  -  \frac{v_\infty}{a} \sum_i \sigma_{v_i}^{-2}\text{sech}^2{\left(\frac{\theta_i - \theta_0}{a}\right)} $  
\item $F_{a a}  =   \frac{v_\infty^2}{a^4} \sum_i  (\theta_i - \theta_0)^2  \sigma_{v_i}^{-2}\text{sech}^4{\left(\frac{\theta_i - \theta_0}{a}\right)} $ 
\item $F_{a v_0}  =  - \frac{v_\infty}{a^2} \sum_i  (\theta_i - \theta_0)  \sigma_{v_i}^{-2}\text{sech}^2{\left(\frac{\theta_i - \theta_0}{a}\right)} $ 
\item $F_{v_0 v_0}  =  \sum_i   \sigma_{v_i}^{-2} $ 
\end{itemize}
\end{document}  